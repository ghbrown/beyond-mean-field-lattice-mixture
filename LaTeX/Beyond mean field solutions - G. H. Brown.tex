\documentclass[10pt]{article}
\usepackage[OT1]{fontenc}
\usepackage[utf8]{inputenc}
\bibliographystyle{plain}

\usepackage{algorithm2e}
\usepackage{amsmath}
\usepackage{amssymb}
\usepackage{gensymb}
\usepackage{mathrsfs}
\usepackage{mathtools}
\usepackage{esint}
\usepackage{braket}
\usepackage{array}
\usepackage{epsfig}
\usepackage{hyperref}
\renewcommand{\baselinestretch}{1.2}
\setlength{\textheight}{9in}
\setlength{\textwidth}{6.5in}
\setlength{\headheight}{0in}
\setlength{\headsep}{0in}
\setlength{\topmargin}{0in}
\setlength{\oddsidemargin}{0in}
\setlength{\evensidemargin}{0in}
\setlength{\parindent}{.3in}


\DeclarePairedDelimiter{\abs}{\lvert}{\rvert} %create the norm sign as symbol
\DeclarePairedDelimiter{\norm}{\lVert}{\rVert} %create the norm sign as symbol

% swap definition of \abs* and \norm* with \abs and \norm
% to scale delimiters without needing to type *
\makeatletter
\let\oldabs\abs
\def\abs{\@ifstar{\oldabs}{\oldabs*}}
\let\oldnorm\norm
\def\norm{\@ifstar{\oldnorm}{\oldnorm*}}
\makeatother

%custom commands here
% \newcommand{\new_command_name}{replacement text}


\begin{document}
\begin{center}
    \textbf{\large Beyond mean field solutions to the lattice mixture model} \\
    Gabriel Brown
\end{center}

%OUTLINE
% statistical thermodynamics review?
% lattice model
% - coordination number, dimension
% - energy of interaction
% - simple results and equations
% - mean field approximation and internal energy formula
% computational approach
% some extremum bounds
%

\section{Lattice model of a mixture} \footnote{The present description of a mixture on a lattice follows closely to \cite{dill}, with minor modifications and my own motivations and explanations.} 
\subsection{Basics}
A mixture of two condensed (liquid or solid) phases $A$ and $B$ may be crudely described by a single-degree-of-freedom-per-site lattice model with uniform coordination number $z$, meaning that each lattice site has $z$ nearest neighbors.
Restricting the energetic interactions to nearest neighbor pairwise, there are three energetic constants for $A-A$, $B-B$, and $A-B$ interactions, respectively: $w_{AA}, \; w_{BB}, \; w_{AB}$, which may be generically positive, negative, or zero.
Let $W$ represent a realization or configuration of the lattice, uniquely specifying the occupation of each site as $A$ or $B$.
The total internal energy $U$ of such a configuration is then given by
\begin{align}
    U(W) = m_{AA} w_{AA} + m_{BB} w_{BB} + m_{AB} w_{AB}
\end{align}
where $m_{XY}$ is the number of $X-Y$ interactions in the system.
Restricting to finite periodic lattice of $N$ total sites with $N_A$ $A$ sites and $N_B$ $B$ sites (uniquely defining a $B$ volume fraction $x = N_B / N$), one can further define the $m_{XY}$s.
Since each pairwise interaction involved two sites, the total number of interactions for a periodic lattice with $N$ sites and coordination $z$ is
\begin{align}
    m_{AA} + m_{BB} + m_{AB} = \frac{z N}{2} = \frac{z}{2} \left( N_A + N_B \right),
\end{align}
further,
\begin{align}
    \frac{z N_A}{2} = m_{AA} + \frac{m_{AB}}{2}, \quad\quad
    \frac{z N_B}{2} = m_{BB} + \frac{m_{AB}}{2}.
\end{align}
With $N$ and $x=N_B/N$ (equivalently $N_A$ and $N_B$) being known system constants, this model is nearly analytical apart from the unknown value of $m_AB$. \footnote{Realistic physical systems are often so large that it is more convenient to work in terms of densities like $N/V$, but since we will be interacting intimately with the lattice description, we maintain use of $N$ directly.} \footnote{One could also treat $m_{AA}$ or $m_{BB}$ as unknowns, the three $m_{\cdot}$ are algebraically related sucht that one of the three uniquely defines the others.}

\section{Computational Implementation}
discuss Metropolis Monte Carlo algorithm (cite MD book)

discuss energy decomposition trick to make each update step

discuss relaxation scheme

discuss forcing every step to exchange A and B

discuss convergence tolerance


\subsection{Approximate solution}
To keep such a lattice model analytical, we use the Bragg-Williams mean field approximation, which posits that the $A$ and $B$ sites are randomly distributed throughout the lattice.
One can then approximate that $A$ sites have on average $z N_B / N$ neighboring $B$ sites, and the corresponding approximation for $m_{AB}$ is
\begin{equation}
    m_{AB} \approx \frac{z N_A N_B}{N} = z N (1-x) x.
\end{equation}
The standard way of writing the total energy of this model is
\begin{align}
    U &=
    \frac{z w_{AA}}{2} N_A +
    \frac{z w_{BB}}{2} N_B +
    k_B T \chi_{AB} \frac{N_A N_B}{N}, \\
    \chi_{AB} &=
    \frac{z}{k_B T} \left( w_{AB} - \frac{w_{AA} + w_{BB}}{2} \right)
\end{align}
where $\chi_{AB}$ is the dimensionless ``exchange parameter''. \footnote{Admittedly, I don't see the utility in artificially introducing temperature dependence to a temperature-independent model via $\chi_{AB}$, but I believe the dimensionless parameter does have some merit for extracting properties from experiment.}
The formula for average energy $\hat{U}=U/N$ per site is then
\begin{align}
    \hat{U} &=
    \frac{z w_{AA}}{2} (1-x) +
    \frac{z w_{BB}}{2} x +
    k_B T \chi_{AB} (1-x) x
\end{align}
which is useful for working with lattices of varying sizes.

We finish the introduction of the lattice model and mean field approximation by noting how the possible errors associated with the Bragg-Williams approximation, which will require a bit of statistical thermodynamics.
The internal energy of a system in a given macrostate (in this case given by system size $N$ and $B$ volume fraction $x$) generically depends on the energies of all the possible microstates $W_i$. \footnote{As an example, for a 1-D system of a binary mixture of macrostate $N=3$ and $x=1/3$ the possible microstates are $[A,A,B],\; [A,B,A],\; [B,A,A]$. In this case, all microstates have the energy, but it should be clear that for larger systems this will generally not be the case.}
Specifically, the energy of a macrostate is given by
\begin{align}
    U = \sum_{i=1}^{N_{micro}} p_i E_i
    = \frac{1}{Q} \sum_{i=1}^{N_{micro}} E_i \exp\left(\frac{-E_i}{k_B T}\right)
\end{align}
where the sum runs over all possible microstates of the macrostate.


\section{Statistical Thermodynamics Review}
The thermodynamic partition function for a macrostate with microscopic degeneracy is
\begin{align}
    Q = \sum_{i=1}^{N_{micro}} W(E_i) \exp \left(\frac{-E_i}{k_B T} \right)
\end{align}
where $W(E_i)$ is the integer degeneracy of all states with energy $E_i$.

The probability of a microstate $j$ is
\begin{align}
    p_j = \frac{\exp\left(\frac{-E_j}{k_B T}\right)}{Q} ,
\end{align}
while relative probabilities of microstates $j$ and $k$ is
\begin{align}
    \frac{p_j}{p_k} =
    \frac{\exp\left(\frac{-E_j}{k_B T}\right)}{\exp\left(\frac{-E_k}{k_B T}\right)}
    = \exp\left(\frac{-(E_j-E_k)}{k_B T}\right).
\end{align}

The energy of a macrostate is
\begin{align}
    U = \sum_{i=1}^{N_{micro}} p_i E_i
    = \frac{1}{Q} \sum_{i=1}^{N_{micro}} E_i \exp\left(\frac{-E_i}{k_B T}\right)
\end{align}

\section{Fitting}
generate a bunch of data, then use nonlinear least squares to fit a new function for $\hat{U}$

do I fit $\hat{U}$ as function of system parameters, or do I fit $m_{AB}/N$ as a function of system parameters?

\newpage
\bibliography{references.bib}
References (USE BIBTEX THOUGH)


\end{document}

